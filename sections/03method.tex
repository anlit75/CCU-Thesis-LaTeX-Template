% !TeX root = ../main.tex

% =================== 研究方法 =================== %
\section{研究方法}

行內方程式($E=mc^2$)有三種方式可以實現,
\begin{verbatim}
    1. \begin{math}E=mc^2\end{math}
    2. $E=mc^2$     % 推薦使用這個格式
    3. \(E=mc^2\)
\end{verbatim}

編號方程式測試,方程式(\ref{equation:eq1});條件方程式測試,方程式(\ref{eq:st});
多行方程式等號對齊測試,方程式(\ref{eq:ht_st})。
(搭配sections/03method.tex來看會更清楚)

\begin{equation}
    \label{equation:eq1}
    \hat{n}=\operatorname*{arg\,max}_{n\in \{1,\dots,M\}}(\mathbf{X}_{n})
\end{equation}

\begin{equation}
    S(t) = P(T > t),\ t \geq 0 \ \
    \begin{cases}
        & \text{if}\ t=0,\ S(t)=1 \\
        & \text{if}\ t=\infty,\ S(t)=0 \\
    \end{cases}
    \label{eq:st}
\end{equation}

\begin{equation}
    \begin{aligned}
        \int_{0}^{t}h(u)du & = -\int_{0}^{t}\frac{d}{du}\log[S(u)]du \\
                           & = -\log[S(t)] + \log[S(0)] \\
                           & = -\log[S(t)] \\
        \Rightarrow\ S(t)  & = \exp\{-\int_{0}^{t}h(u)du\}
    \end{aligned}
    \label{eq:ht_st}
\end{equation}

無編號方程式測試,方程式:
$$ \frac{P\left((t \leq T < t + \Delta t) \cap (T \geq t)\right)}{P(T \geq t)} = \frac{P(t \leq T < t + \Delta t)}{P(T \geq t)} $$

在方程式環境中,若要使用粗體請用\textbackslash bm\{\},文字請用\textbackslash text\{\},左右括號希望越外層越大請使用\textbackslash left(以及\textbackslash right),如下:
$$ \bm{h}^{(k)}_u = \sigma\left(\bm{W}^{(k)}_{\text{self}}\bm{h}^{(k-1)}_u + \bm{W}^{(k)}_{\text{neigh}}\sum_{v \in \mathcal{N}(u)}\bm{h}^{(k-1)}_v + b^{(k)}\right) $$


演算法測試,演算法\ref{algorithm:alg1}。Require和Ensure可以改成Input和Output,如演算法\ref{alg:alg2}。

\begin{algorithm}[!htb]
    \caption{範例演算法}
    \label{algorithm:alg1}
    \begin{algorithmic}[1]
        \Require
        The set of positive samples for current batch, $P_n$;
        The set of unlabelled samples for current batch, $U_n$;
        Ensemble of classifiers on former batches, $E_{n-1}$;
        \Ensure
        Ensemble of classifiers on the current batch, $E_n$;
        \State Extracting the set of reliable negative and/or positive samples $T_n$ from $U_n$ with help of $P_n$;
        \label{code:fram:extract}
        \State Training ensemble of classifiers $E$ on $T_n \cup P_n$, with help of data in former batches;
        \label{code:fram:trainbase}
        \State $E_n=E_{n-1}cup E$;
        \label{code:fram:add}
        \State Classifying samples in $U_n-T_n$ by $E_n$;
        \label{code:fram:classify}
        \State Deleting some weak classifiers in $E_n$ so as to keep the capacity of $E_n$;
        \label{code:fram:select}
        \Return $E_n$;
    \end{algorithmic}
\end{algorithm}

\begin{algorithm}[!htb]
    \caption{範例演算法2}
    \label{alg:alg2}
    \renewcommand{\algorithmicrequire}{\textbf{Input:}}
	\renewcommand{\algorithmicensure}{\textbf{Output:}}
    \begin{algorithmic}[1]
        \Require{Somthing}
        \For{each Somthing}
            \If{Somthing area is less than 90\%}
                \State Discard
            \Else
                \State Extract Somthing
            \EndIf
        \EndFor
        \Ensure{Somthing Cool}
    \end{algorithmic}
\end{algorithm}

\clearpage

% ------------------- 流程圖 -------------------- %
\subsection{流程圖}

\begin{figure}[!htb]
    \centering
    \begin{tikzpicture}[node distance=10pt]
        \node[draw, rounded corners]                        (start)   {Start};
        \node[draw, below=of start]                         (step 1)  {Step 1};
        \node[draw, below=of step 1]                        (step 2)  {Step 2};
        \node[draw, diamond, aspect=2, below=of step 2]     (choice)  {Choice};
        \node[draw, right=30pt of choice]                   (step x)  {Step X};
        \node[draw, rounded corners, below=20pt of choice]  (end)     {End};

        \draw[->] (start)  -- (step 1);
        \draw[->] (step 1) -- (step 2);
        \draw[->] (step 2) -- (choice);
        \draw[->] (choice) -- node[left]  {Yes} (end);
        \draw[->] (choice) -- node[above] {No}  (step x);
        \draw[->] (step x) -- (step x|-step 2) -> (step 2);
    \end{tikzpicture}
    \caption{範例流程圖}
    \label{figure:flowchart1}
\end{figure}


% ------------------- 小標題 -------------------- %
\subsection{小標題}

化學結構式測試,圖\ref{figure:chem1},電路圖測試,圖\ref{figure:circ1}。

\begin{figure}[!htb]
    \centering
    \chemfig{
        H_3C-[:72]{\color{blue}N}*5(-
        *6(-(={\color{red}O})-
        {\color{blue}N}(-CH_3)-
        (={\color{red}O})-
        {\color{blue}N}(-CH_3)-=)--
        {\color{blue}N}=-)
    }
    \caption{範例化學結構式}
    \label{figure:chem1}
\end{figure}

\begin{figure}[!htb]
    \centering
    \begin{circuitikz}
        \draw (0,0) to[V=1V] (0,2)
            to[R=$1\Omega$] (2,2) -- (4,2)
            to[C=1F] (4,0) -- (0,0);
        \draw (2,2) to[L=1H, *-*] (2,0);
    \end{circuitikz}
    \caption{範例電路圖}
    \label{figure:circ1}
\end{figure}


% ------------------- 程式碼 -------------------- %
\subsection{程式碼}

程式碼測試,程式碼\ref{code:python}。支援的程式語言可以參考\ \href{https://en.wikibooks.org/wiki/LaTeX/Source_Code_Listings#Supported_languages}{Source Code Listings}。

\begin{lstlisting}[language=Python , caption={範例程式碼}, label={code:python}]
    import os

    def main():
        print("Hello World!")

    if __name__ == "__main__":
        main()
\end{lstlisting}
