% !TeX root = ../main.tex

% =================== 研究方法 =================== %
\section{研究方法}

在段落中使用行內數學公式可以 \begin{math}E=mc^2\end{math},可以$E=mc^2$,也可以\(E=mc^2\)。
方程式測試,方程式\ref{equation:eq1}。演算法測試,演算法\ref{algorithm:alg1}。

\begin{equation}[htbp]
    \label{equation:eq1}
    (\hat{n})=\operatorname*{arg\,max}_{n\in \{1,\dots,M\}}(\mathbf{X}_{n})
\end{equation}

\begin{algorithm}[htbp]
    \caption{範例演算法}
    \label{algorithm:alg1}
    \begin{algorithmic}[1]
        \Require
        The set of positive samples for current batch, $P_n$;
        The set of unlabelled samples for current batch, $U_n$;
        Ensemble of classifiers on former batches, $E_{n-1}$;
        \Ensure
        Ensemble of classifiers on the current batch, $E_n$;
        \State Extracting the set of reliable negative and/or positive samples $T_n$ from $U_n$ with help of $P_n$;
        \label{code:fram:extract}
        \State Training ensemble of classifiers $E$ on $T_n \cup P_n$, with help of data in former batches;
        \label{code:fram:trainbase}
        \State $E_n=E_{n-1}cup E$;
        \label{code:fram:add}
        \State Classifying samples in $U_n-T_n$ by $E_n$;
        \label{code:fram:classify}
        \State Deleting some weak classifiers in $E_n$ so as to keep the capacity of $E_n$;
        \label{code:fram:select} \\
        \Return $E_n$;
    \end{algorithmic}
\end{algorithm}

% ------------------- 流程圖 -------------------- %
\subsection{流程圖}

\begin{figure}[htbp]
    \centering
    \begin{tikzpicture}[node distance=10pt]
        \node[draw, rounded corners]                        (start)   {Start};
        \node[draw, below=of start]                         (step 1)  {Step 1};
        \node[draw, below=of step 1]                        (step 2)  {Step 2};
        \node[draw, diamond, aspect=2, below=of step 2]     (choice)  {Choice};
        \node[draw, right=30pt of choice]                   (step x)  {Step X};
        \node[draw, rounded corners, below=20pt of choice]  (end)     {End};
        
        \draw[->] (start)  -- (step 1);
        \draw[->] (step 1) -- (step 2);
        \draw[->] (step 2) -- (choice);
        \draw[->] (choice) -- node[left]  {Yes} (end);
        \draw[->] (choice) -- node[above] {No}  (step x);
        \draw[->] (step x) -- (step x|-step 2) -> (step 2);
    \end{tikzpicture}
    \caption{範例流程圖}
    \label{figure:flowchart1}
\end{figure}

% ------------------- 小標題 -------------------- %
\clearpage
\subsection{小標題}

化學結構式測試,圖\ref{figure:chem1},電路圖測試,圖\ref{figure:circ1}。

\begin{figure}[htbp]
    \centering
    \chemfig{
        H_3C-[:72]{\color{blue}N}*5(- 
        *6(-(={\color{red}O})-
        {\color{blue}N}(-CH_3)-
        (={\color{red}O})-
        {\color{blue}N}(-CH_3)-=)--
        {\color{blue}N}=-)
    }
    \caption{範例化學結構式}
    \label{figure:chem1}
\end{figure}

\begin{figure}[htbp]
    \centering
    \begin{circuitikz}
        \draw (0,0) to[V=1V] (0,2)
            to[R=$1\Omega$] (2,2) -- (4,2)
            to[C=1F] (4,0) -- (0,0);
        \draw (2,2) to[L=1H, *-*] (2,0);
    \end{circuitikz}
    \caption{範例電路圖}
    \label{figure:circ1}
\end{figure}